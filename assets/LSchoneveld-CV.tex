 % LaTeX resume using res.cls

\documentclass[margin]{res}


%\usepackage{helvetica} % uses helvetica postscript font (download helvetica.sty)

%\usepackage{newcent}   % uses new century schoolbook postscript font 

\setlength{\textwidth}{5in} % set width of text portion


\begin{document}
	
	
	% Center the name over the entire width of resume:
	
	\moveleft.5\hoffset\centerline{\large\bf Liam Schoneveld}
	
	% Draw a horizontal line the whole width of resume:
	
	\moveleft\hoffset\vbox{\hrule width\resumewidth height 1pt}\smallskip
	
	% address begins here
	
	% Again, the address lines must be centered over entire width of resume:
	
    \moveleft.5\hoffset\centerline{liam.sch@gmail.com  \hspace{28em}  +33 7 78 02 41 71}
	
	% \moveleft.5\hoffset\centerline{ }
	
	\begin{resume}
		
		
		\section{EDUCATION} {\bf Master of Science (Artificial Intelligence)} {\sl Cum Laude} \hfill 2015-2017\\
		The University of Amsterdam\\
		{\sl IRP: Semi-Supervised Learning with Generative Adversarial Networks} (2017)
		
		{\bf Bachelor of Commerce (Liberal Studies)} \hfill 2009-2013\\
		The University of Sydney\\
		{\sl Majors in Economics and Econometrics}
		
		\section{PROFESSIONAL \\ EXPERIENCE}
		
		{Principal AI Researcher,} {\bf Powder} (\$14.5m 2021 Series A) \hfill Oct 2019-Present\vspace{2mm}\newline
        Lead Powder's AI team; help define and implement the AI strategy:
		\vspace{1mm}
		\begin{itemize}
			\item Designed and developed model training and deployment pipelines for three key product features: detecting which videogame is being played, detecting highlights or exciting moments in the gameplay footage, and verifying the absence of adult content. These models run in real-time, and on-device.
			\item Co-manage the day-to-day workflow of Powder's seven-person AI team; ensure effective intra- and cross-team communication.
			\item Research collaboration with Dr.  Alice Othmani at Universit\'e Paris-Est Cr\'eteil (UPEC); first author on two published papers (see Publications), both achieving state-of-the-art results on popular emotion recognition benchmark datasets.
			\item Research into self-supervised learning for highlights detection led to an order of magnitude reduction in the amount of labels required for training models.
			\item Technical topics include: deep learning for object detection, classification, video highlighting, facial expression recognition, MLops, deployment to edge devices, model quantisation and pruning, self-supervised learning, etc.
		\end{itemize}
		
		{Data Scientist (Deep Learning),} {\bf Pandascore} (\$6m 2020 Series A) \hfill 2017-2019
		\vspace{1mm}
		\begin{itemize}
			\item Developing and deploying computer vision models for real-time events and stats detection in e-sports tournaments.
			\item Training and deploying machine learning models using these computer vision-derived data to predict match outcomes in real time.
			\item Key projects:
			\begin{itemize}
				\item Detecting hero positions on the League of Legends minimap in real time (plus auto-retraining and deployment when new heroes are released).
				\item Probabilistic inference to calculate and optimize betting odds in real time over all possible outcomes in an ongoing Overwatch match.
				\item Using GANs to synthetically generate training data of new playable characters in Overwatch before any `real' training data was available.
			\end{itemize}
			\item Contributing to growth, product strategy and recruitment efforts.
		\end{itemize}
		
		{Machine Learning Specialist,} {\bf Scyfer (acquired by Qualcomm)} \hfill Feb-Sep 2017\vspace{1mm}\newline
		Completed my Artificial Intelligence (AI) masters' thesis under the supervision of Taco Cohen and Max Welling while interning at Scyfer, an AI consultancy acquired by Qualcomm in late 2017. Focus was on researching and developing semi-supervised deep learning techniques, with an emphasis on computer vision.
		
		{Econometrics Contractor,} {\bf LEK Consulting} \hfill 2016-2017\vspace{1mm}\newline
		Contracted to develop econometric revenue analysis and forecasting models for several assets belonging to a large (ASX top 50) Australian client of LEK.
		
		{Economics Consultant,} {\bf PricewaterhouseCoopers Australia} \hfill 2013-2016\vspace{1mm}\newline
		Consulted with clients primarily from the transport industry. Focus was on econometric and economic modelling (e.g. demand forecasting, cost-benefit analyses, project appraisal, etc.) and associated analysis and reporting.
		
		{Software Developer,} {\bf Law in Order Pty Ltd} \hfill 2010-2013\vspace{1mm}\newline
		Produced electronic databases of legal evidence for Law in Order's clients. Proposed, developed and maintained a quality assurance software tool that continued to be used at Law in Order many years after my departure.
		
		\section{PUBLICATIONS}
		
		Schoneveld, L. \& Othmani A. (2021).
		\textit{Towards a General Deep Feature Extractor for Facial Expression Recognition}. 28th IEEE International Conference on Image Processing (ICIP), 2021, pp. 2339-2342.
		
		Schoneveld, L., Othmani, A., \& Abdelkawy, H. (2021). \textit{Leveraging Recent Advances in Deep Learning for Audio-Visual Emotion Recognition}. Pattern Recognition Letters, 146, pp. 1-7.
		
		\section{PROGRAMMING\\LANGUAGES}
		\begin{itemize}
			\item Python data science / ML stack (numpy, sklearn, pandas, matplotlib, etc.)
			\item Deep learning libraries (Pytorch, Tensorflow, Keras, Theano)
			\item MLops, ML pipeline reproducability, ML model deployment to multiple frameworks (Apple CoreML, TensorRT, tflite, MLFlow, DVC, Docker, Comet.ml)
			\item Cloud providers: experience with GCP, AWS, Azure, Paperspace
			\item Unix/Linux, SQL, Git, R, Matlab, C++, Java, HTML, CSS, JavaScript, Julia.
		\end{itemize}
		
		\section{OTHER \\ ACHIEVEMENTS}
		
		\begin{itemize}
			\item Ongoing: Maintain a blog with posts on machine learning at \textbf{nlml.github.io}
			\item Ongoing: Contribute to open source projects on GitHub - e.g. Pytorch, mlflow
%			\item Ongoing: Answer questions on StackOverflow, see: stackoverflow.com/u/6167850
			\item 2021: Presented at poster session of IEEE ICIP (online event)
			\item 2018: Presented at Meetup Computer Vision Paris
			\item 2018: Participated in the 2018 Amsterdam Dance Event Hackathon (worked in a team to build a deep learning-based reverse image search application)
			\item 2016: Placed 1st out of 50 student groups in a Kaggle-style machine learning competition as part of the UvA/VU course \textit{Data Mining Techniques}
			\item 2015: Placed 17th out of 985 participants in Kaggle's {\sl Facebook Recruiting IV: Human or Robot?} data science competition
			\item 2013: Placed 2nd out of 108 students in the \textit{Operations Management} course at the University of Sydney
			\item 2010: Awarded two separate International Exchange Scholarships by the University of Sydney, based on academic merit
			\item 2008: Placed 7th of 2730 students in Australian Higher School Certificate course {\sl Software Design and Development.}
		\end{itemize}
	\textbf{Hobbies} include music (listening and playing keyboard/drums), surfing, climbing, cycling, and traveling.
		
	\end{resume}
	
\end{document}